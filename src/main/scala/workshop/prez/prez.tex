\documentclass{beamer}
\usepackage{graphicx}
\usepackage[hypcap]{caption}
\usepackage{xcolor}
\usepackage{listings}
\usepackage[most, listings]{tcolorbox}
\usepackage{ulem, contour}
\usepackage{booktabs}
\usepackage{multicol, multicolrule}



% ---------------------------------------------------------------------------- %
%                                    COLORS                                    %
% ---------------------------------------------------------------------------- %
\definecolor{ScalaRed}{HTML}{DD332F}
\definecolor{Titleblue}{RGB}{114, 146, 162}

\definecolor{black}{rgb}{0,0,0}
\definecolor{green}{rgb}{0,0.5,0}
\definecolor{red}{rgb}{1,0,0}
\definecolor{blue}{rgb}{0,0,1}

\definecolor{codeGreen}{rgb}{0,0.6,0}
\definecolor{codeGray}{rgb}{0.5,0.5,0.5}
\definecolor{codePurple}{rgb}{0.58,0,0.82}
\definecolor{codeBgColor}{rgb}{0.95,0.95,0.95}


% ---------------------------------------------------------------------------- %
%                                  CODE BLOCKS                                 %
% ---------------------------------------------------------------------------- %
\lstdefinestyle{scalaCode}{
  % backgroundcolor=\color{white},
  basicstyle=\footnotesize,
  breakatwhitespace=false,
  breaklines=true,
  % captionpos=b,
  commentstyle=\color{codeGreen},
  % escapeinside=
  extendedchars=true,
  % firstnumber=1,
  % frame=tb,
  % framerule=0.25pt,
  keepspaces=true,
  keywordstyle=\color{blue},
  language=scala,
  gobble=6,
  morekeywords={@, given, using, targetName, infix, transparent, inline, *, extension},
  % numbers=left,
  % numbersep=5pt,
  % numberstyle=\tiny\color{black},
  % rulecolor=\color{black},
  showspaces=false,
  showstringspaces=false,
  showtabs=false,
  stepnumber=1,
  stringstyle=\color{codePurple},
  tabsize=2,
}
\lstset{style=scalaCode}


% ---------------------------------------------------------------------------- %
%                                 Configuration                                %
% ---------------------------------------------------------------------------- %

\graphicspath{{./img}}

\SetMCRule{extend-fill=false}
\usetheme{Madrid}
\usecolortheme{beaver}
\useinnertheme{circles}
\setbeamercolor{itemize item}{fg=ScalaRed}
\setbeamercolor{item projected}{bg=ScalaRed}
% \setbeamertemplate{enumerate items}[default]
% \setbeamertemplate{navigation symbols}{}
\setbeamercovered{transparent}
\setbeamercolor{local structure}{fg=ScalaRed}
\useoutertheme{smoothtree}

\renewcommand{\ULdepth}{1.8pt}
\contourlength{0.8pt}
\captionsetup{labelfont={it, bf}, textfont={it}}
\urlstyle{same}

\AtBeginSection{
  \begin{frame}
    \frametitle{Outline}
    \tableofcontents[currentsection]
  \end{frame}
}
\AtBeginSubsection{
  \begin{frame}
    \frametitle{Outline}
    \tableofcontents[currentsection, currentsubsection]
  \end{frame}
}

\tcbset{listing engine=listings}



% ---------------------------------------------------------------------------- %
%                                   COMMANDS                                   %
% ---------------------------------------------------------------------------- %
\newcommand{\ul}[1]{%
	\uline{\phantom{#1}}%
	\llap{\contour{white}{#1}}%
}

\newcommand{\uhref}[2]{%
  \ul{\href{#1}{#2}}%
}


\author[]{Lucas Nouguier}
\institute[]{Polytech Montpellier}
\date{13 November 2023}
\title{Functional Design \& Parser combinators}

\begin{document}
\titlegraphic{\includegraphics[scale=0.1]{scala-spiral.png}}
\frame{\titlepage}

\section{Introduction}
\begin{frame}[fragile]
  \frametitle{ADTs}

  An Algrebraic Data Type is a composite type made of other types

  \begin{example}
    \begin{lstlisting}
      type List[A]   = Nil | Cons[A]
      type Option[A] = None | Some[A]


      sealed trait Tree
      case class Name() extends Tree
      case class Select(qual: Tree, name: Name) extends Tree
      case class Method(fun: Tree, args: ArgClauses) extends Tree
      case class If(p: Tree, thenp: Tree, elsep: Tree) extends Tree

      case class ArgClauses(args: List[List[Tree]]) extends Tree
    \end{lstlisting}
  \end{example}
\end{frame}

\begin{frame}
  \frametitle{Failure handling}

  \begin{table}[h]
    \centering
    \begin{tabular}{cccccc}
      \toprule
       &                 & \textbf{Null}s & \textbf{Exception}s & \textbf{ADT}s & \\
      \midrule
       & Difficulty      & Easy           & Easy                & Less easy     & \\
       & Failure reason  & Unknown        & Known               & Known         & \\
       & Performances    & Best           & Bad                 & Better        & \\
       & Expressivity    & Bad            & Better              & Best          & \\
       & Runtime failure & Possible       & Possible            & No            & \\
    \end{tabular}
  \end{table}

  \begin{table}[h]
    \centering
    \begin{tabular}{rcccc}
      \textbf{Benchmark}   & \textbf{Mode} & \textbf{Count} & \textbf{Score} & \textbf{Error} \\
      \midrule
      No exceptions        & avgt          & 10             & 0.046          & ± 0.003        \\
      Throw \& catch       & avgt          & 10             & 16.268         & ± 0.239        \\
      Throw \& no catch    & avgt          & 10             & 17.874         & ± 3.199        \\
      Throw w/o stacktrace & avgt          & 10             & 1.174          & ± 0.014        \\
      \bottomrule
    \end{tabular}
    \caption{Comparison \& Benchmarks (ms/op)}
  \end{table}
\end{frame}

\begin{frame}[fragile]
  \frametitle{Parser combinators}

  Create a complex parser from simple ones

  \begin{example}
    \begin{lstlisting}
      val tpeSep: Parser[String]  // ( : ) | (: ) | ( :)
      val alphas: Parser[Char]    // [a-zA-Z]
      val name: Parser[String]    = alphas.repeat.combineAll

      // Parser[Param] -- x: Int
      (name <* tpeSep).product(name).map(???)

      // Parser[Queue[Param]] -- // x: A, y: B, z: C
      (paramParser <* paramSepParser).repeat
        .combineWith(Queue.empty)(_ :+ _)
        .product(paramParser.orElse(Parser.empty))
        .map(???)
    \end{lstlisting}
  \end{example}
\end{frame}

\begin{frame}[fragile]
  \frametitle{Parser combinators --- example}

  \begin{overprint}
    \onslide<1>
    Given the following code, which parser can we create?
    \begin{example}
      \begin{lstlisting}
        package scalala
  
        object Main:
          val x = 1
          val y = 2
          val adder = (a: Int, b: Int) => a + b
  
          def main(args: Array[String]): Unit =
            println(foo(x, y))
  
          def foo(x: Int, y: Int) = x + y
      \end{lstlisting}
    \end{example}

    \onslide<2>
    \begin{table}
      \centering
      \begin{tabular}{rll}
        \toprule
        \textbf{Parser}    & \textbf{Description}  & \textbf{Example}                                    \\
        \midrule
        \texttt{name}      & A simple name         & \texttt{x}, \texttt{y}, \texttt{main}, \texttt{foo} \\
        \texttt{literal}   & A literal value       & \texttt{1}, \texttt{2}                              \\
        \texttt{val}       & A value definition    & \texttt{val x = 1}                                  \\
        \texttt{param}     & A method parameter    & \texttt{a:Int}, \texttt{args:Array[String]}         \\
        \texttt{paramList} & A parameter list      & \texttt{a:Int, b:Int}                               \\
        \texttt{method}    & A method definition   & \texttt{main(args:Array[String]) = }                \\
        \texttt{function}  & A function definition & \texttt{adder = (a:Int, b:Int) => }                 \\
        \texttt{class}     & A class definition    & \texttt{object Main}                                \\
        \texttt{source}    & A source file         & \texttt{package scalala; ...}                       \\
        \bottomrule
      \end{tabular}
    \end{table}

    \onslide<3>
    \includegraphics[width=0.96\textwidth]{img/parser-structure.png}
  \end{overprint}
\end{frame}

\begin{frame}[fragile]
  \frametitle{General approach}

  \begin{overprint}
    \onslide<1>
    With \(\mathcal{P}\) the parser algrebra and \(\mathcal{A}\) the ADT algebra. I'll use reification to implement this parser, with 3 kinds of methods:
    \begin{itemize}
      \item Constructors \(c\): \(\forall x \notin \mathcal{P}, c(x) \in \mathcal{P}\)
      \item Combinators \(f\): \(\forall c_1, c_2 \in \mathcal{P}, f(c_1, c_2) \in \mathcal{P}\)
      \item Interpreters \(z\): \(\forall c \in \mathcal{P}, x \notin \mathcal{P}, z(c, x) \in \mathcal{A}\)
    \end{itemize}
  
    \begin{example}[Reification methods]
      \begin{lstlisting}[gobble=8]
        // Constructor
        def string(s: String): Parser[String]
        // Combinator
        def orElse[A, B](p1: Parser[A], p2: Parser[B]): Parser[A | B]
        // Interpreter
        Parser[A].parse(input: String): A
      \end{lstlisting}
    \end{example}
    
    \onslide<2>
    For each abstract concept, we'll have a concrete implementation (with a \texttt{case class}) and reification methods. For instance, for a simple string parser and a simple mapper:
    
    \begin{lstlisting}
      trait Parser[+A]:
        def map[B](f: A => B): Parser[B] = ParserMap(this, f)
      object Parser:
        def string(s: String): Parser[String] = ParserString(s)
    
      case class ParserString(s: String) extends Parser[String]
      case class ParserMap[A, B](
        source: Parser[A],
        f: A => B
      ) extends Parser[B]
    \end{lstlisting}
  \end{overprint}
\end{frame}

\section{Type classes --- Functional design}
\subsection{Basics}

\begin{frame}
  \frametitle{by-value vs by-name}

  % TODO

\end{frame}

\begin{frame}
  \frametitle{lazy vs eager}

  % TODO

\end{frame}

\begin{frame}[fragile]
  \frametitle{Given/using (implicit values)}

  \texttt{using} clause defines a value to be injected by the compiler, based on the expected type among the \texttt{given} values

  \begin{example}[Given/using]
    \begin{lstlisting}
      given life: Int = 42
      def foo(using i: Int): Int = i
  
      val meaningOfLife1 = foo              // : Int = 42
      // rewritten as foo(life) by the compiler
      val meaningOfLife2 = foo(using life)  // : Int = 42
      // we can also explicitly pass the value
    \end{lstlisting}
  \end{example}
\end{frame}

\begin{frame}[fragile]
  \frametitle{Variance}

  \begin{definition}[Variance]
    Describe the relation between generic types. Given a generic type \(\mathcal{F}\):
    \begin{description}
      \item[Invariant] \(\forall A,B\quad A \neq B \Leftrightarrow \mathcal{F}[A] \neq \mathcal{F}[B]\)
      \item[Covariant] \(\forall A,B\quad A <: B \Leftrightarrow \mathcal{F}[A] <: \mathcal{F}[B]\)
      \item[Contravariant] \(\forall A,B\quad A <: B \Leftrightarrow \mathcal{F}[B] <: \mathcal{F}[A]\)
    \end{description}
  \end{definition}

  It allows a more flexible design, but has some constraints for type-safety. For a covariant type, we cannot use it's type param as method param type

  \begin{example}[Covariance constraint]
    \begin{lstlisting}
      class Foo[+A]:
        def foo[B >: A](x: B) = ??? // cannot use A as param type
    \end{lstlisting}
  \end{example}

\end{frame}

\subsection{Type classe}

\begin{frame}[fragile]
  \frametitle{Usual OOP way}

  \begin{example}[Inheritance]
    \begin{lstlisting}
      trait Encoder      { def encode: String }
      trait Combiner[A] { def combine(a: A): String }

      abstract class Animal extends Encoder
        with Combiner[Animal]

      case class Cat() extends Animal:
        override def encode(): String
        override def combine(b: Animal): String
      case class Dog() extends Animal:
        override def encode(): String
        override def combine(b: Animal): String
    \end{lstlisting}
  \end{example}
\end{frame}

\begin{frame}[fragile]
  \frametitle{Better design}

  \begin{example}[Composition]
    \begin{lstlisting}
      trait Encoder[A]  { def encode (a: A): String }
      trait Combiner[A] { def combine(a: A, b: A): String }

      abstract class Animal(
        encoder: Encoder[Animal],
        combiner: Combiner[Animal]
      ):
        def encode = encoder.encode(this)
        def combine(b: Animal) = combiner.combine(this, b)
    \end{lstlisting}
  \end{example}
\end{frame}

\begin{frame}[fragile]
  \frametitle{Functional way}

  \begin{example}[Type class]
    \begin{lstlisting}
      trait Encoder[-A] { def encode (a: A): String }

      abstract class Animal

      extension [A](a: A)(using encoder: Encoder[A])
        def encode: String = encoder.encode(a)

      given catEncoder: Encoder[Cat] with
        def encode(a: Cat) = "A cat"

      aCat.encode // "A cat"
      // rewritten as encode(aCat, catEncoder)
    \end{lstlisting}
  \end{example}
\end{frame}

\begin{frame}
  \frametitle{Type class definition}

  \begin{enumerate}
    \item Define a trait (the behavior)
    \item Define your methods
    \item Define your trait instances
    \item (Optional) redefine methods as extension methods
  \end{enumerate}
\end{frame}

\begin{frame}[fragile]
  \frametitle{Exercices}

  \begin{enumerate}
    \item Define a type class \texttt{JsonEncoder} for a \texttt{case class Person} with a name, an age and an address
    \item Create a \texttt{JsonEncoder} for a \texttt{List[T]}
    \item Create a \texttt{JsonEncoder} for an \texttt{Option[T]}
    \item Try it with a \texttt{List[Option[Person]]}
  \end{enumerate}

  Given syntax help
  \begin{lstlisting}[gobble=4]
    given Type with
      // trait methods implementation
    given [A]: Type[A] with
      // trait methods implementation
    given [A](using otherGiven: OtherType[A]): Type[A] with
      // trait methods implementation
  \end{lstlisting}
\end{frame}

\begin{frame}[fragile]
  \frametitle{Exercices solution --- Part 1}

  \begin{lstlisting}[gobble=4]
    case class Person(...)
    
    // 1. TC definition
    trait JsonEncoder[-T]:
      def encode(t: T): String 

    // 3. TC instance for Person
    given JsonEncoder[Person] with
      def encode(person: Person): String = ???

    // 4. Redefine methods as extension methods
    extension [T](t: T)(using encoder: JsonEncoder[T])
      def toJson: String = encoder.encode(t)
  \end{lstlisting}

\end{frame}

\begin{frame}[fragile]
  \frametitle{Exercices solution --- Part 2}

  \begin{lstlisting}[gobble=4]
    // TC instance for List[T]
    given [T: JsonEncoder]: JsonEncoder[List[T]] with
      def encode(list: List[T]): String =
        list.map(_.toJson).mkString("[", ",", "]")

    // TC instance for Option[T]
    given [T: JsonEncoder]: JsonEncoder[Option[T]] with
      def encode(option: Option[T]): String =
        option.map(_.toJson).getOrElse("null")

    List(Some(lulu), None, Some(zozo)).toJson
    // : String = [
    //  {"name":"lulu",...}
    //  null
    //  {"name":"zozo",...}
    //]
  \end{lstlisting}
\end{frame}

\subsection{Common type classes}

\begin{frame}[fragile]
  \frametitle{Base types}

  \begin{definition}[Parser \& Result ADTs]
    \begin{lstlisting}
      sealed trait Parser[+A]:
        def parse(input: String): Result[A]
        protected def parse(input: String, index: Int): Result[A]
  
      sealed trait Result[+A]:
        def map[B](f: A => B): Result[B]
        def orElse[B](that: => Result[B]): Result[A | B]

      case class Success[+A](...) extends Result[A]
      case class Failure extends Result[Nothing]
    \end{lstlisting}
  \end{definition}
\end{frame}

% --------------------------------- Functors --------------------------------- %
\begin{frame}[fragile]
  \frametitle{Functor}

  \textbf{Problem} Need to transform a value inside any kind of (unrelated) data structure

  % \onslide<2->
  \begin{definition}[Functor]
    \begin{lstlisting}
      trait Functor[F[_]]:
        def map[A, B](fa: F[A])(f: A => B): F[B]
    \end{lstlisting}
  \end{definition}

  % \onslide<3->
  \begin{itemize}
    \item Single abstraction for any \ul{generic} type
          \begin{itemize}
            \item Valid types: \texttt{Functor[List]}, \texttt{Functor[Option]}\dots
            \item Invalid types: \texttt{Functor[Int]}, \texttt{Functor[Person]}\dots
          \end{itemize}
    \item A bit more verbose (can be hidden with extension methods)
    \item Easy to switch from a data structure to another
  \end{itemize}
\end{frame}

\begin{frame}[fragile]
  \frametitle{Functor --- map implementation}

  \begin{enumerate}
    \item New concept \({\rightarrow}\) new case class
          \begin{itemize}
            \item Input: \texttt{Parser[A]} \& \texttt{A => B}
            \item Ouput: \texttt{Parser[B]}
          \end{itemize}
    \item No constructors, one combinator \& interpreter
  \end{enumerate}

  \begin{lstlisting}[gobble=4]
    trait Parser[+A]:
      def map[B](f: A => B): Parser[B] = ParserMap(this, f)

    final case class ParserMap[A, B](
      source: Parser[A],
      f: A => B
    ) extends Parser[B]:
      override def parse(input: String, index: Int): Result[B] =
        source.parse(input, index).map(f)
  \end{lstlisting}
\end{frame}

% ---------------------------------- Monoids --------------------------------- %
\begin{frame}[fragile]
  \frametitle{Monoid}

  \textbf{Problem} Need to combine any kind of values

  % \onslide<2->
  \begin{definition}[Monoid]
    \begin{lstlisting}
      trait Monoid[A]:
        def empty: A
        def combine(a: A, b: A): A
    \end{lstlisting}
  \end{definition}

  % \onslide<3->
  \begin{itemize}
    \item Single abstraction for \ul{any} type
    \item Can serve as a AND or OR operation
          \begin{itemize}
            \item AND:\ \texttt{combine(Parser1,Parser2) = Parser1 then Parser2}
            \item OR:\ \texttt{combine(Parser1,Parser2) = Parser1 orElse Parser2}
          \end{itemize}
    \item \texttt{Monoid}s without \texttt{empty} are called \texttt{Semigroup}s
  \end{itemize}
\end{frame}

\begin{frame}[fragile]
  \frametitle{Monoid --- orElse implementation}

  \begin{enumerate}
    \item New concept \({\rightarrow}\) new case class
          \begin{itemize}
            \item Input: \texttt{Parser[A]} \& \texttt{Parser[B]}
            \item Output: \texttt{Parser[A | B]}
          \end{itemize}
    \item No constructors, one combinator \& interpreter
  \end{enumerate}

  \begin{lstlisting}[gobble=4]
    trait Parser[+A]:
      def orElse[B](that: => Parser[B]): Parser[A | B] =
        ParserOrElse(this, that)

    final class ParserOrElse[A, B](
      left: Parser[A],
      right: => Parser[B]
    ) extends Parser[A | B]:
      def parse(input: String, index: Int): Result[A | B] =
        left.parse(input, index)
            .orElse(right.parse(input, index))
  \end{lstlisting}
\end{frame}

% ------------------------------- Semigroupals ------------------------------- %
\begin{frame}[fragile]
  \frametitle{Semigroupal}

  \textbf{Problem} Need to operate on multiple values at once, without combining them (e.g. to instanciate a class)

  \begin{definition}[Semigroupal]
    \begin{lstlisting}
      trait Semigroupal[F[_]]:
        def product[A, B](fa: F[A], fb: F[B]): F[(A, B)]
    \end{lstlisting}
  \end{definition}

  \begin{itemize}
    \item Single abstraction for \ul{generic} type
    \item Different from \texttt{Semigroup}
    \item Merge two values into one to use later
  \end{itemize}
\end{frame}

\begin{frame}[fragile]
  \frametitle{Semigroupal --- product implementation}

  \begin{enumerate}
    \item New concept \({\rightarrow}\) new case class
          \begin{itemize}
            \item Input: \texttt{Parser[A]} \& \texttt{Parser[B]}
            \item Output: \texttt{Parser[(A, B)]}
          \end{itemize}
    \item No constructors, one combinator \& interpreter
  \end{enumerate}

  \begin{lstlisting}[gobble=4]
    trait Parser[+A]:
      def product[T >: A, B](that: Parser[B]): Parser[(T, B)] =
        ParserProduct(this, that)

    final case class ParserProduct[A, B](
      left: Parser[A],
      right: Parser[B]
    ) extends Parser[(A, B)]:
      def parse(input: String, index: Int): Result[(A, B)] =
        left.parse(input, index) match
          case fail: Failure => fail
          case Success(leftResult, _, offset) =>
            right.parse(input, offset).map((leftResult, _))
  \end{lstlisting}
\end{frame}

\begin{frame}[fragile]
  \frametitle{Semigroupal --- Monadic Combinations}

  When the container is a \texttt{Monad}, the \texttt{.product()} does a cartesian product (and bypasses `invalid' values like \texttt{None})

  \begin{example}[Monadic combination]
    \begin{lstlisting}
      Semigroupal[List].product(
        List(1, 2),
        List(4, 5),
      ) // List((1, 4), (1, 5), (2, 4), (2, 5))
      
      Semigroupal[Future].product(
        Future(throw new Exception),
        Future(throw new RuntimeException)
      ) // Failure(java.lang.Exception)
    \end{lstlisting}
  \end{example}
  /!\textbackslash{} With \texttt{List} \& \texttt{Future} being \texttt{Monad}s /!\textbackslash{}
\end{frame}

\begin{frame}[fragile]
  \frametitle{Semigroupal --- Applicative Combinations}

  When the container is not a \texttt{Monad}, the \texttt{.product()} keeps all the values (allows errors accumulation for instance)
  \begin{example}[Applicative combination]
    \begin{lstlisting}
      Semigroupal[???].product(
        Validated.invalid(List("Badness")),
        Validated.invalid(List("Fail"))
      ) // Invalid(List(Badness, Fail))

      Semigroupal[???].product(
        Validated.valid("Good"),
        Validated.valid(47),
      ) // Valid("Good", 47)
    \end{lstlisting}
  \end{example}
  /!\textbackslash{} With \texttt{Validated} \ul{not} being a \texttt{Monad} /!\textbackslash{}
\end{frame}

% ------------------------------- Applicatives ------------------------------- %
\begin{frame}[fragile]
  \frametitle{Applicative}

  \textbf{Problem} Need to apply a independent effects to value(s) of a container

  \begin{definition}[Applicative definition]
    \begin{lstlisting}
      trait Applicative[F[_]]:
        def pure[A](a: A): F[A]
        def ap[A, B](ff: F[A => B])(fa: F[A]): F[B]
    \end{lstlisting}
  \end{definition}

  \begin{itemize}
    \item Single abstraction for \ul{generic} type
    \item Gives access to the \texttt{mapN} method, easier to manipulate than \texttt{product}
  \end{itemize}
\end{frame}

\begin{frame}[fragile]
  \frametitle{Applicative --- Example}

  Reminder: we cannot use \texttt{.flatMap}, it is not defined
  \begin{example}[Usage]
    \begin{lstlisting}
      val f: (Int, Int) => Int = _ + _
      val intList1 = List(5, 10, 15)
      val intList2 = List(0, 1)

      val adder = intList1.map(i1 => (i2: Int) => f(i1, i2))
      // List(i2 => f(5, c), i2 => f(10, c), i2 => f(15, c))
      adder.ap(intList2)
      // List(5, 6, 10, 11, 15, 16)
    \end{lstlisting}
  \end{example}
\end{frame}

% ---------------------------------- Monads ---------------------------------- %
\begin{frame}[fragile]
  \frametitle{Monad}

  \textbf{Problem} Need to chain operations on a same-kind container

  \begin{definition}[Monad definition]
    \begin{lstlisting}
      trait Monad[F[_]]:
        def flatMap[A, B](fa: F[A])(f: A => F[B]): F[B]
    \end{lstlisting}
  \end{definition}

  \begin{itemize}
    \item Single abstraction for any \ul{generic} type
    \item Continue the chain only on success
    \item Stop the chain on first failure
  \end{itemize}
\end{frame}

\begin{frame}[fragile]
  \frametitle{Monad --- flatMap implementation}

  \begin{enumerate}
    \item New concept \({\rightarrow}\) new case class
          \begin{itemize}
            \item Input: \texttt{Parser[A]} \& \texttt{A => Parser[B]}
            \item Output: \texttt{Parser[B]}
          \end{itemize}
    \item No constructors, one combinator \& interpreter
  \end{enumerate}

  \begin{lstlisting}[gobble=4]
    trait Parser[+A]:
      def flatMap[B](f: A => Parser[B]): Parser[B] =
        ParserFlatMap(this, f)

    final case class ParserFlatMap[A, B](
      source: Parser[A],
      f: A => Parser[B]
    ) extends Parser[B]:
      def parse(input: String, index: Int): Result[B] =
        source.parse(input, index) match
          case fail: Failure => fail
          case Success(result, input, offset) =>
            f(result).parse(input, offset)
  \end{lstlisting}

\end{frame}

% ---------------------------------- Summary --------------------------------- %
\begin{frame}
  \frametitle{Summary}

  \begin{description}
    \item[Functor] transform a value inside a container
    \item[Semigroupal] tuple values from any containers
    \item[Applicative] apply a independent effects to value(s) of a container
    \item[Monad] chain operations on same-kind containers
    \item[Monoid] combine values
  \end{description}
\end{frame}

\begin{frame}[fragile]
  \frametitle{Type class hierarchy}

  \begin{figure}[h]
    \centering
    \includegraphics[width=0.9\linewidth]{img/tc-hierarchy.png}
    \caption{Hierarchy}
  \end{figure}
\end{frame}
\section{Parsing expressions}
\subsection{Binary expressions}

\begin{frame}[fragile]
  \frametitle{Binary expression ADT}

  \begin{definition}
    \begin{lstlisting}
      sealed trait Expr:
        def +(that: Expr): Expr
  
      case class Num(value: Int) extends Expr
      case class Var(name: String) extends Expr
      case class Add(left: Expr, right: Expr) extends Expr
    \end{lstlisting}
  \end{definition}
\end{frame}

\begin{frame}[fragile]
  \frametitle{Addition parser}
  \begin{overprint}
    \onslide<1>
    Given the following parsers:
  
    \begin{multicols}{2}
      \begin{description}
        \item[number] [0--9]+
        \item[variable] [a-zA-Z]+
        \item[plus] [ ]*+[ ]*
        \item[expr] variable \textbar{} number
      \end{description}
    \end{multicols}
    \begin{lstlisting}
      val parser0: Parser[Add] =
        (expr, plus, expr).mapN((l, _, r) => l + r)
      parser0.parse("x + 1 + a").get // Add(Var("x"), Num(1))
    \end{lstlisting}
    \begin{lstlisting}
      val parser1: Parser[Add] =
        (expr, plus, parser1 orElse expr).mapN((l, _, r) => l + r)
      parser1.parse("x + 1 + a").get // NullPointerException
    \end{lstlisting}
    \begin{lstlisting}
      def parser2: Parser[Add] =
        (expr, plus, parser2 orElse expr).mapN((l, _, r) => l + r)
      parser2.parse("x + 1 + a").get // StackOverflowError
    \end{lstlisting}
    \onslide<2>
    \textbf{Solution} Delay the parser's evaluation
    \begin{lstlisting}
      // In Parser.scala
      object Parser:
        def lzy[A](parser: => Parser[A]) = Delayed(parser)

      class Delayed[A](p: => Parser[A]) extends Parser[A]:
        lazy val cached = p
        def parse(input: String, index: Int): Result[A] =
          cached.parse(input, index)
    \end{lstlisting}
    \begin{lstlisting}
      // Addition parser
      val parser3: Parser[Add] =
        (expr, plus, lzy(parser3) orElse expr).mapN((l,_,r) => l+r)

      parser3.parse("x + 1 + a").get
      // : Add = Add(Var("x"), Add(Num(1), Var("a")))
    \end{lstlisting}
  \end{overprint}
\end{frame}

\begin{frame}
  \frametitle{Json parser}

  You can now try to implement a JSON parser\@:D

\end{frame}
\end{document}
